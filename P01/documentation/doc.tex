\documentclass[12pt]{article}
\usepackage{polski}
\usepackage{lmodern}
\usepackage[utf8]{inputenc}
\usepackage{tabto}
\usepackage{indentfirst} %pierwszy akapit posiada wcięcie
\title{Interpolacja wielomianowa - projekt}
\author{Natalia Wojtania i Grzegorz Chojnacki}
%\date{}
\begin{document}
\maketitle

\section{Zadanie}
\subsection{Tytuł}
Tytuł zadania to "Dwutlenek węgla".
\subsection{Treść}
Program,  który  oszacuje  tempo  przyrostu dwutlenku węgla w atmosferze Ziemi. Węzły mają  przedstawiać  ilość  wyemitowanego  do atmosfery $ CO_{2}$ w  ciągu  roku  lub  w  innym przedziale czasowym.
\subsection{Metoda}
W programie należy wykorzystać metodę Newtona.
\subsubsection{Opis metody}
Mając zadany układ punktów $\{(x_{j},y_{j}),j=0,1,2,3...,n\}$, gdzie $x_{0},x_{1},x_{2},...,x_{n}$ są węzłami interpolacyjnymi, a $y_{0},y_{1},...,y_{n}$ wartościami, poszukujemy wielomianu interpolacyjnego $P \in \sqcap_{n}$ spełniającego warunki $P(x_{i})=y_{i}, i=0,1,2,...,n$ w postaci :
$$ P(x)=b_{0}+b_{1}(x-x_{0})+b_{2}(x-x_{0})(x-x_{1})+...+b_{n}(x-x_{0})\cdot...\cdot(x-x_{n-1}).$$
Z wyżej wymienionych warunków otrzymamy układ z niewiadomymi $b_{0},b_{1},...,b_{n}.$ 
Z pierwszego równania $P(x_{0})=y_{0}=b_{0}$, następnie $P(x_{1})=y_{1}=b_{0}+b_{1}(x-x_{0})$, stąd $b_{1}=\frac{y_{1}-y_{0}}{x_{1}-x_{0}}$ itd.
\subsubsection{Przykład}
\begin{tabular}{c|c|c|c}
$x_{i}$&0&2&3 \\ \hline
$y_{i}$&1&11&19
\end{tabular}
\hspace{10mm} $ P(x)=b_{0}+b_{1}(x-0)+b_{2}(x-0)(x-2)$
\\ \\
Z warunku $P(0)=1$ mamy $ b_{0}=1,$ z $P(2)=11$ mamy $b_{1}=5,$ z $ P(3)=19 $ mamy $ b_{2}=1.$
\\Stąd $P(x)=1+5(x-0)+1(x-0)(x-2)=x^{2}+3x+1.$
\section{Opis implementacji algorytmu}
Implementacja realizująca metodę Newtona.
\subsection{Dane wejściowe}
Na wejściu program pobiera od użytkownika wartości punktów \\$P_{j}(x,y), j=0,1,2,...,n$, gdzie 'Rok pomiarów' to x, a 'Przyrost $ CO_{2}$ [t]' to y. Realizacja wprowadzenia danych możliwa jest na dwa sposoby. Poprzez bezpośrednie wpisanie wartości lub zaimportowanie danych z pliku JSON.
\subsection{Struktury danych}
Każdy wielomian to lista współczynników, gdzie poszczególny indeks odpowiada potędze $x$ przy danym współczynniku. Zerowy element obrazuje $ x^{0} $($x$ do potęgi zerowej), pierwszy element odpowiada $ x^{1} $ itd. 
\begin{description}
 
 \item[PRZYKŁAD]:\hspace{5 mm}  [2,-1,2,-2] oznacza $-2x^{3} + 2x^{2} - x + 2$.
 \\
\end{description}
Program korzysta ze struktury klasy Polynomial, która wspiera operacje:
\begin{itemize}
 \item Dodawanie: Przewidziane dla wyrazów wolnych jak i wielomianów. Odpowiednie elementy list obu wielomianów są dodawane.
 \begin{description}
 
 \item[PRZYKŁAD]:\hspace{5 mm}[0, 0, 3] + [1, 2, 0, 4] = [1, 2, 3, 4] \\ oznacza 
  $3x^2 + 4x^{3}+2x+1 = 4x^{3}+3x^{2}+2x+1 $
   \end{description}
 \item Mnożenie: Podczas mnożenia razy $x^{n}$ ($x$ do potęgi n-tej) każdy wykładnik jest podnoszony o potęgę $n$, a lista przesuwa się o $n$ miejsc. W przypadku mnożenia wielomianu razy wielomian, drugi wielomian mnożymy przez każdy składnik pierwszego i sumujemy.
 \begin{description}
 
\item[PRZYKŁAD]:\hspace{5 mm}[3, 3] $\cdot$ [-3, 3] = [-9,0,9] \\ oznacza $(3x + 3)(3x - 3) = 9x^{2} - 9$
 \end{description}
 \item Wyliczanie wartości w punkcie: Za pomocą metody Hornera.
\end{itemize}

\subsection{Funkcje pomocnicze}

W programie wykorzystywane są takie funkcje jak:
\begin{enumerate}
\item map: transformacja każdego elementu danej listy 
\item reduce: sprowadzenie listy do pojedynczej wartośći, przy pomocy funkcji działającej na kolejnych elementach danej listy 
\item filter: filtrowanie danych w liście, spełniających określony warunek
\end{enumerate}
\subsection{Przebieg działania}
Program wyświetla komunikat: 'Wprowadź listę punktów poniżej'. Jeśli zostały wprowadzone prawidłowe dane, to na bieżąco wyświetlany jest odpowiedni wielomian. 
Próba ręcznego wprowadzenia nieprawidłowych danych, które weryfikowane są w programie poprzez funkcję getPoints() skutkuje brakiem wyświetlenia wielomianu. Podobnie dzieje się w sytuacji pozostawienia pustego pola.
NADPISYWANIE X??????
\par Dane dostarczone z pliku JSON program sprawdza poprzez funkcję parsePoints() oraz wyświetla komunikat "Błąd wczytywania pliku" w przypadku niepowodzenia.\\
\par Następnie funkcja recalculate() zajmuje się przekazaniem punktów, w celu dalszego rachunku, a także wyświetleniem wyniku.\\
Funkcja getPolynomial() klasy NewtonEvaluator zwraca wielomian, licząc wcześniej niewiadome b0, b1...; mając tylko jeden punkt zwracana jest od razu wartość y. W przeciwnym wypadku zwracany jest wyliczany wielomian. 
\\
Tworzona jest tablica z niewiadomymi b0,b1,...,bn
Później liczona jest grupa 




W najbardziej kluczowych przypadkach wykorzystywane jest programowanie dynamiczne.
\\ \\
Wynikiem działania programu jest wielomian interpolacyjny obrazujący oszacowanie tempa przyrostu dwutlenku węgla.
%file:///C:/Users/HP/AppData/Local/Temp/W1_Interpolacja%20wielomianowa.pdf
\end{document}