\documentclass[12pt]{article}
\usepackage{polski}
\usepackage{lmodern}
\usepackage[utf8]{inputenc}
\usepackage{tabto}
\usepackage{indentfirst} %pierwszy akapit posiada wcięcie
\usepackage{graphicx}
\title{Aproksymacja równań różniczkowych - projekt}
\author{Natalia Wojtania i Grzegorz Chojnacki}
%\date{}
\begin{document}
\maketitle

\section{Zadanie}
\subsection{Tytuł}
Tytuł zadania to ,,Metody numeryczne dla zagadnieńróżniczkowych".
\subsection{Treść}
Napisz program, który rozwiąże trzema metodami(Eulera, zmodyfikowaną Eulera oraz Heuna)zagadnienie różniczkowe: $$y'(x)= f(x,y(x)), y(1)=5,$$ gdzie $$ f(x,y(x))=-5x^4+2 \sqrt{y+x^5-5}+5.$$
Program ma również obliczyć dokładność dla każdej z tych metod, porównując je z dokładnym rozwiązaniem:$$ y(x)=-x^5+x^2+5.$$
\subsection{Metody}
W programie należy wykorzystać metodę Eulera, zmodyfikowaną metodę Eulera oraz metodę Heuna (udoskonaloną metodę Eulera).
\subsubsection{Opis metod}
!!!!!
Metoda siecznych (interpolacji liniowej) polega na przyjęciu, że funkcja ciągła na dostatecznie małym odcinku w przybliżeniu zmienia się w sposób liniowy. Można wtedy na odcinku $[a,b]$ krzywą $y=f(x)$ zastąpić sieczną. Za przybliżoną wartość pierwiastka przyjmuje się punkt przecięcia siecznej z osią odciętych $OX$. Miejsce przecięcia tej prostej z osią $X$ jest przybliżonym wynikiem szukanego miejsca zerowego, o ile różnica bezwzględna wartości z dwóch ostatnich iteracji jest mniejsza od założonej dokładności.  Metoda ta wymaga ustalenia na przedziale $[a,b]$ dwóch punktów startowych $x_0$ i $x_1$.\\
Metodę siecznych dla funkcji $f(x)$, mającej pierwiastek w przedziale $[ a , b ]$ można zapisać następującym wzorem rekurencyjnym:

 $$x_{k+1}=x_k - \frac{f(x_k)(x_k-x_{k-1})}{f(x_k)-f(x_{k-1})}, k \geq 1$$ i $$x_0=a, x_1=b, $$ gdzie w każdym kroku $ x_{k+1}$ to miejsce zerowe siecznej wykresu $y=f(x)$ w punktach $(x_{k-1},f(x_{k-1}))$ oraz$ (x_{k},f(x_{k})) $, czyli prostej $$y=\frac{f(x_k)-f(x_{k-1})}{x_k-x_{k-1}}(x-x_k)+f(x_k)$$
\subsubsection{Przykład}
Dla zagadnienia różniczkowego $y'(x)= f(x,y(x)), y(1)=5,$  na przedziale [1,2] przyjmując $N = 2$ policzyć rozwiązanie trzema metodami oraz obliczyć dokładność dla każdej z metod, wiedząc, że dokładne rozwiązanie to\\ $y(x)=-x^5+x^2+5$ oraz         $f(x,y(x))=-5x^4+2 \sqrt{y+x^5-5}+5.$


\section{Opis implementacji algorytmu}
Implementacja realizująca metody Eulera, zmodyfikowaną Eulara oraz Heuna.
\subsection{Dane wejściowe}
Na wejściu program pobiera od użytkownika liczbę $n$, określająca podział odcinka $[a, b]$ oraz liczbę $b$, informująca o końcu odcinka $[a, b]$.



\subsection{Przebieg działania}
!!!
Program wyświetla komunikat: ,,Wprowadź dokładność rozwiązania $\epsilon \in(0,1)$ ". Jeśli została wprowadzona prawidłowa wartość dokładności, to program poprzez funkcję \emph{calculate} wylicza przybliżone rozwiązanie i dzięki funkcji \emph{refresh} wyświetla je wraz z liczbą kroków.
Próba wprowadzenia nieprawidłowych danych, które weryfikowane są w programie w funkcji \emph{refresh} skutkuje wyświetleniem stosownego ostrzeżenia.
\par Następnie funkcja \emph{calculate} klasy \emph{SecantMethod} zajmuje się wyliczeniem przybliżonego rozwiązania w oparciu o podaną dokładność i przedział wyszukany za pomocą funkcji \emph{findInterval}. Szukanie przedziału zaczyna się od $[0, 1]$ i jeżeli funkcja nie przechodzi w nim przez oś $OX$, to przedział jest poszerzany.\\
Funkcja \emph{getNext}, której argumentami są $a$ i $b$ odpowiednio oznaczające $x_{k-1}$ oraz $x_{k}$ zwraca wartość poszczególnego $x_{k+1}$.
\\
Funkcja \emph{isGoodEnough} sprawdza czy różnica  $|x_k - x_{k-1}|$ jest mniejsza od podanej przez użytkownika dokładności. Jeśli tak, to kończymy przekazując wynik oraz ilość kroków. W przeciwnym wypadku liczone jest kolejne przybliżenie tak długo, aż warunek zostanie spełniony.


Wynikiem działania programu jest przybliżone rozwiązanie równania: $x_k$ oraz liczba wykonanych kroków: $k$.
\newpage
\subsection{Najważniejsze fragmenty programu}
!!!
secantMethod.js
\begin{verbatim}

class SecantMethod {
  f = x => 4*x**3 + 5*x**2 + 6*x - 7

  constructor(precision) {
    this.precision = precision
    this.interval = this.findInterval()
  }

  findInterval = (a = 0, b = 1) => this.f(a) * this.f(b) < 0
    ? [a, b]
    : this.findInterval(a - (b - a), b + (b - a))

  // a = x_{k-1}, b = x_{k}
  getNext = (a, b) => b - (this.f(b) * (b - a)) / (this.f(b) - this.f(a))
  isGoodEnough = (next, prev) => Math.abs(next - prev) < this.precision

  calculate() {
    const g = (a, b, steps = 2) => {
      const next = this.getNext(a, b)
      return this.isGoodEnough(next, b)
        ? ({ result: next, steps })
        : g(b, next, steps + 1)
    }
    return g(this.interval[0], this.interval[1])
  }
}

\end{verbatim}
\newpage
gui.js
\begin{verbatim}


const gui = new (class {
  input  = document.getElementById('input')
  result = document.getElementById('result')
  steps  = document.getElementById('steps')
  error  = document.getElementById('error')

  refresh() {
    const precision = this.getPrecision()
    if (0 < precision && precision < 1) {
      this.clearError()
      const answer = new SecantMethod(precision).calculate()
      this.result.innerText = answer.result
      this.steps.innerText  = answer.steps
    } else this.setError()
  }

  setError()   { this.error.innerText  = 'Wprowadzona wartość poza przedziałem (0, 1)' }
  clearError() { this.error.innerText  = '' }

  update = debounce(() => this.refresh(), 10)

  getPrecision = () => Number.parseFloat(this.input.value)

})()
\end{verbatim}
\newpage
\subsection{Widok działania programu}
\begin{figure}[h]
\centering
\includegraphics[scale=0.65]{correctData.jpg}
\caption{Prawidłowo wprowadzone dane}
\end{figure}

\begin{figure}
\centering
\includegraphics[scale=0.65]{wrongData.jpg}
\caption{Nieprawidłowo wprowadzone dane}
\end{figure}

\end{document}
